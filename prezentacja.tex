\documentclass[10pt, aspectratio=169]{beamer}

\usepackage[utf8]{inputenc}
\usepackage[T1]{fontenc}
\usepackage[polish]{babel}
\usepackage{booktabs}
\usepackage{graphicx}
\usepackage{tikz}
\usetikzlibrary{shadows}
\usepackage{fontawesome5}
\usepackage{caption}

\usetheme[progressbar=frametitle, block=fill]{metropolis}

% Definicja palety kolorów "Vibrant Tech"
\definecolor{DeepBlue}{HTML}{0F2027}   % Ciemny, elegancki granat
\definecolor{Highlight}{HTML}{FF0055}   % Neonowy róż (akcent)
\definecolor{SoftGray}{HTML}{F0F2F5}    % Tło bloków

% Przypisanie kolorów do elementów Beamera
\setbeamercolor{normal text}{fg=DeepBlue, bg=white}
\setbeamercolor{alerted text}{fg=Highlight}
\setbeamercolor{frametitle}{bg=DeepBlue, fg=white}
\setbeamercolor{progress bar}{fg=Highlight, bg=gray!20}
\setbeamercolor{itemize item}{fg=Highlight} % Kolor kropek listy

% --- STYL BLOKÓW (ZAOKRĄGLONE) ---
\usepackage{tcolorbox}
\tcbuselibrary{skins}

% Redefinicja standardowych bloków (np. Hipoteza)
\setbeamertemplate{blocks}[rounded][shadow=true]
\setbeamercolor{block title}{bg=DeepBlue, fg=white}
\setbeamercolor{block body}{bg=SoftGray, fg=black}

% Redefinicja bloków alert (np. Luka rynkowa)
\setbeamercolor{block title alerted}{bg=Highlight, fg=white}
\setbeamercolor{block body alerted}{bg=red!5, fg=black}

% --- TŁO SLAJDÓW (DESIGN) ---
\setbeamertemplate{background canvas}{
    \begin{tikzpicture}[remember picture,overlay]
        \shade[top color=white, bottom color=gray!5] (current page.north west) rectangle (current page.south east);
        \fill[Highlight] (current page.south east) circle (20pt);
        \fill[DeepBlue] (current page.north west) circle (10pt);
    \end{tikzpicture}
}

% --- DANE TYTUŁOWE ---
\title{HRflow: Inteligentna Platforma HR}
\subtitle{Synteza Koncepcji (SRS) i Plan Sprintu 1 (PoC)}
\author{}
\institute{Zarządzanie Projektem Informatycznym - prezentacja podsumowująca}
\date{}

\begin{document}

% --- SLAJD TYTUŁOWY ---
{
\setbeamertemplate{background canvas}[default]
\setbeamercolor{background canvas}{bg=DeepBlue}
\begin{frame}[plain]
    \transboxout[duration=1]
    \centering
    \color{white}
    \Huge \textbf{HRflow} \\
    \large \textcolor{Highlight}{Inteligentna Platforma HR}
    \vspace{1cm}
    
    \normalsize
    Synteza Koncepcji (SRS) i Plan Sprintu 1 (PoC)
    \vspace{1.5cm}
    
    \small \textbf{Zespół Projektowy:} \\
    Adrian Jabłoński, Paweł Gorgolewski, \\
    Kamil Pierzchała, Łukasz Bartoszek, Bartosz Balawender
\end{frame}
}

% =============================================================================
% CZĘŚĆ I: KONCEPCJA
% =============================================================================
\section{Część I: Koncepcja Biznesowa}

% -----------------------------------------------------------------------------
% CELE PROJEKTU
% -----------------------------------------------------------------------------
\begin{frame}{1. Główny Cel i KPI}
    \transdissolve[duration=0.5]
    
    \textbf{Wizja Produktu:}
    Automatyzacja cyklu życia pracownika w MŚP, pozwalająca skupić się na ludziach, a nie na procesach.

    \vspace{0.5cm}
    \textbf{Kluczowe Wskaźniki Sukcesu (KPI):}
    \begin{itemize}
        \setlength\itemsep{0.8em}
        \item[\faIcon{clock}] \textbf{Redukcja Time-to-Hire:} Skrócenie czasu rekrutacji z 45 do \textbf{20 dni}.
        \item[\faIcon{bullseye}] \textbf{Trafność Selekcji:} Wzrost wskaźnika akceptacji na rozmowę (IAR) o +5 p.p.
        \item[\faIcon{shield-alt}] \textbf{Bezpieczeństwo:} 100\% automatyzacja odbierania dostępów.
        \item[\faIcon{smile}] \textbf{Candidate Experience:} Osiągnięcie wskaźnika cNPS +50.
    \end{itemize}
\end{frame}

% -----------------------------------------------------------------------------
% REALIZACJA CELÓW
% -----------------------------------------------------------------------------
\begin{frame}{2. Jak technologia realizuje cele?}
    \transpush[direction=90]
    \small
    \begin{columns}[T, onlytextwidth]
        \column{0.48\textwidth}
            \begin{block}{\faIcon{cogs} Kluczowe Funkcjonalności}
                \begin{itemize}
                    \item \textbf{Semantic Matching Engine (NLP):}
                    Analiza CV pod kątem kompetencji pochodnych.
                    \textit{$\rightarrow$ Cel: Szybsza selekcja.}
                    \item \textbf{Portal Kandydata:}
                    Tracking statusu aplikacji na żywo.
                    \textit{$\rightarrow$ Cel: Wzrost cNPS.}
                    \item \textbf{Auto-Offboarding:}
                    Integracja z AD do blokady kont.
                    \textit{$\rightarrow$ Cel: Bezpieczeństwo.}
                \end{itemize}
            \end{block}
        
        \column{0.48\textwidth}
            \begin{alertblock}{\faIcon{tachometer-alt} Atrybuty Jakościowe}
                \begin{itemize}
                    \item \textbf{Wydajność:} Cache Redis (czas < 3s).
                    \item \textbf{RODO:} Szyfrowanie AES-256.
                    \item \textbf{Skalowalność:} Docker.
                \end{itemize}
            \end{alertblock}
    \end{columns}
\end{frame}

% --- GANTT CHART ---
\begin{frame}{Harmonogram Realizacji (Roadmapa)}
    \transblindsvertical[duration=0.7]
    \begin{figure}
        \centering
        \includegraphics[width=0.95\textwidth, height=0.75\textheight, keepaspectratio]{images/gantt_chart.png}
        \caption{Plan MVP rozpisany na 3 miesiące (styczeń - kwiecień 2026).}
    \end{figure}
\end{frame}

% -----------------------------------------------------------------------------
% INNOWACYJNOŚĆ
% -----------------------------------------------------------------------------
\begin{frame}{3. Innowacyjność (USP)}
    \transglitter[direction=315]
    \begin{alertblock}{Luka Rynkowa}
        Brak zaawansowanych narzędzi dostępnych cenowo dla sektora MŚP.
    \end{alertblock}

    \vspace{0.5cm}
    \textbf{Nasza przewaga:}
    \begin{itemize}
        \item[\faIcon{brain}] \textbf{Demokratyzacja AI:} Matching semantyczny dla średnich firm.
        \item[\faIcon{link}] \textbf{Holistyczne Podejście:} Połączenie dokumentów, IT i onboardingu.
        \item[\faIcon{search-plus}] \textbf{Technologia:} NLP zamiast słów kluczowych.
    \end{itemize}
\end{frame}

% =============================================================================
% CZĘŚĆ II: SPRINT 1 (POC)
% =============================================================================
\section{Część II: Plan Sprintu 1 (PoC)}

% --- BACKLOG ---
\begin{frame}{Zakres Prac: Product Backlog}
    \transblindshorizontal[duration=0.5]
    \begin{figure}
        \centering
        \includegraphics[width=0.95\textwidth, height=0.75\textheight, keepaspectratio]{images/backlog.png}
        \caption{Priorytetyzacja zadań w Jira na najbliższe sprinty.}
    \end{figure}
\end{frame}

% -----------------------------------------------------------------------------
% HIPOTEZA I MODEL
% -----------------------------------------------------------------------------
\begin{frame}{1. Hipoteza i Model Matematyczny}
    \transsplitverticalout
    \begin{columns}[T, onlytextwidth]
        \column{0.55\textwidth}
            \begin{block}{Hipoteza Badawcza}
                Algorytm NLP dostarczy wynik trafności zgodny z oceną rekrutera w \textbf{minimum 80\%} przypadków.
            \end{block}
            \vspace{0.5cm}
            \textbf{Model (Baseline):}
            \begin{equation*}
                Score = \left( \sum w_i \cdot s_i \right) - P
            \end{equation*}
        
        \column{0.40\textwidth}
            \begin{figure}
                \centering
                % Dodajemy cień pod obrazkiem dla efektu 3D
                \begin{tikzpicture}
                    \node[drop shadow={opacity=0.5}, inner sep=0pt] {
                        \includegraphics[width=\textwidth, height=0.7\textheight, keepaspectratio]{images/jira_us_scoring.png}
                    };
                \end{tikzpicture}
                \caption{\scriptsize Szczegóły User Story}
            \end{figure}
    \end{columns}
\end{frame}

% -----------------------------------------------------------------------------
% EKSPERYMENT
% -----------------------------------------------------------------------------
\begin{frame}{2. Eksperyment: Kroki w Sprincie}
    \transwipe[direction=0]
    \begin{enumerate}
        \item \textbf{Setup:} Wybór biblioteki NLP (SpaCy vs NLTK).
        \item \textbf{Prototyp API:} Endpoint JSON (CV + Oferta).
        \item \textbf{Dane Testowe:} 10 Person (CV idealne, słabe, błędne).
        \item \textbf{Walidacja:} Porównanie rankingu System vs Rekruter.
    \end{enumerate}
\end{frame}

% --- SPRINT BOARD ---
\begin{frame}{Realizacja: Tablica Sprintu 1}
    \transboxin[duration=1]
    \begin{figure}
        \centering
        \includegraphics[width=0.95\textwidth, height=0.75\textheight, keepaspectratio]{images/sprint_board.png}
        \caption{Postęp prac badawczych i wdrożeniowych (Jira Board).}
    \end{figure}
\end{frame}

% -----------------------------------------------------------------------------
% KRYTERIA SUKCESU
% -----------------------------------------------------------------------------
\begin{frame}{3. Kryteria Sukcesu (DoD dla PoC)}
    \transdissolve
    Eksperyment uznamy za udany, jeśli:
    \vspace{0.3cm}
    \begin{itemize}
        \setlength\itemsep{0.8em}
        \item[\faCheckCircle] \textbf{Trafność:} Algorytm wskaże to samo TOP 3 co człowiek.
        \item[\faCheckCircle] \textbf{Wydajność:} Czas oceny CV < 3 sekundy.
        \item[\faCheckCircle] \textbf{Semantyka:} Rozpoznawanie relacji (np. React $\rightarrow$ JS).
        \item[\faCheckCircle] \textbf{Odporność:} Obsługa błędów formatowania PDF.
    \end{itemize}
\end{frame}

% -----------------------------------------------------------------------------
% SCENARIUSZE DECYZYJNE
% -----------------------------------------------------------------------------
\begin{frame}{4. Scenariusze Decyzyjne}
    \transglitter[direction=90]
    \begin{columns}[T, onlytextwidth]
        \column{0.48\textwidth}
            % Użycie tcolorbox dla ładniejszych kolorowych ramek
            \begin{tcolorbox}[colback=green!10!white, colframe=green!60!black, title=\faIcon{thumbs-up} \textbf{Scenariusz Pozytywny (GO)}]
                \textit{Trafność > 80\%}
                \begin{itemize}
                    \item Budowa pełnego API i bazy.
                    \item Frontend: wizualizacja wyników.
                    \item Dostrajanie wag.
                \end{itemize}
            \end{tcolorbox}

        \column{0.48\textwidth}
            \begin{tcolorbox}[colback=orange!10!white, colframe=orange!80!red, title=\faIcon{random} \textbf{Scenariusz Negatywny (PIVOT)}]
                \textit{Trafność < 60\%}
                \begin{itemize}
                    \item \textbf{Plan B:} Rezygnacja z NLP.
                    \item Model oparty na tagach.
                    \item Rozwój manualnej checklisty.
                \end{itemize}
            \end{tcolorbox}
    \end{columns}
\end{frame}

% --- ZAKOŃCZENIE ---
{
\setbeamertemplate{background canvas}[default]
\setbeamercolor{background canvas}{bg=DeepBlue}
\begin{frame}[plain]
    \transboxin
    \centering
    \color{white}
    \vspace{1cm}
    \huge \textbf{Dziękujemy za uwagę.}
    \vspace{1cm}
    
    \normalsize
    Gotowi do Sprintu 1.
    \vspace{1.5cm}
    
    \footnotesize
    \faIcon{question-circle} Czas na pytania.
\end{frame}
}

\end{document}Zespół Projektowy